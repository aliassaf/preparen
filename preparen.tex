\documentclass{article}

\usepackage{preparen}

\def\arraystretch{1.2}

\title{Preparen --- A \LaTeX package for pretty-printing precedence parentheses}

\author{Ali Assaf}

\date{2014/10/14 version 1.1}

\begin{document}

\maketitle

\begin{abstract}
This package provides commands for pretty-printing parentheses according to 
precedence levels. It can be used to define and write abstract syntax trees in
\LaTeX without worrying about parentheses.
\end{abstract}

\section{Commands}

Pretty-printing in this package is based on a notion of \emph{precedence
levels} that dictate the order in which operations should be parsed relative
to each other.
The idea is that when we switch from a lower precedence level to a higher
precedence level (say a multiplication inside an addition), we don't need to
print parentheses because the expression would not be ambiguous, such as in:
\[
1 + 2 \times 3
\]
When we switch from a higher level to a lower level (an addition inside a
multiplication), we need to print parentheses to disambiguate, such as in:
\[
1 \times \left(2 + 3\right)
\]

The package provides two commands \verb|\requirelvel| and \verb|\protectlevel|.
The first takes a number and an expression, and requires the precedence level
inside the expression to be at least as high as the given number. The second
takes a number and an expression and compares the given level to the current
required level. If it is smaller than the current required level, it surrounds
the expression with parentheses:
\begin{verbatim}
\[
1 \times \requirelevel{2}{
  \protectlevel{1}{2 + 3}}
\]
\end{verbatim}
\[
1 \times \requirelevel{2}{
  \protectlevel{1}{2 + 3}}
\]
If it is higher than or equal to the current required level, it prints the
expression unchanged:
\begin{verbatim}
\[
1 + \requirelevel{1}{
  \protectlevel{2}{2 \times 3}}
\]
\end{verbatim}
\[
1 + \requirelevel{1}{
  \protectlevel{2}{2 \times 3}}
\]


\section{Examples}

\subsection{Arithmetic expressions}

Arithmetic expressions are given by the following non-amgiguous grammar:
\[
\begin{array}{lll}
A & ::= & A + M \mid A - M \mid M \\
M & ::= & M \times V \mid M \div V \mid V \\
V & ::= & x \mid \left(A\right) \\
\end{array}
\]
We can assign the addition and subtraction operators a precedence level of 1
and the multiplication and division operators a precedence level of 2
(variables have a precedence level of 3 but we don't need to give them an
explicit operator).
\begin{verbatim}
\newcommand{\plusop}[2]{
  \protectlevel{1}{
    \requirelevel{1}{#1} + \requirelevel{2}{#2}}}

\newcommand{\minusop}[2]{
  \protectlevel{1}{
    \requirelevel{1}{#1} - \requirelevel{2}{#2}}}

\newcommand{\timesop}[2]{
  \protectlevel{2}{
    \requirelevel{2}{#1} + \requirelevel{3}{#2}}}

\newcommand{\divop}[2]{
  \protectlevel{2}{
    \requirelevel{2}{#1} + \requirelevel{3}{#2}}}
\end{verbatim}
We can now use these constructs to write arbitrary arithmetic expressions
without worrying about parentheses.

\newcommand{\plusop}[2]{
  \protectlevel{1}{
    \requirelevel{1}{#1} + \requirelevel{2}{#2}}}

\newcommand{\minusop}[2]{
  \protectlevel{1}{
    \requirelevel{1}{#1} - \requirelevel{2}{#2}}}

\newcommand{\timesop}[2]{
  \protectlevel{2}{
    \requirelevel{2}{#1} \times \requirelevel{3}{#2}}}

\newcommand{\divop}[2]{
  \protectlevel{2}{
    \requirelevel{2}{#1} \div \requirelevel{3}{#2}}}

\begin{verbatim}
\[
\minusop{\plusop{1}{\timesop{2}{\plusop{x}{3}}}}{\minusop{4}{5}}
\]
\end{verbatim}
\[
\minusop{\plusop{1}{\timesop{2}{\plusop{x}{3}}}}{\minusop{4}{5}}
\]

\subsection{$\lambda$-calculus expressions}

Expressions of the $\lambda$-calculus are given by the following non-amgiguous
grammar:
\[
\begin{array}{lll}
L & ::= & \lambda x.L \mid A \\
A & ::= & A\,V \mid V \\
V & ::= & x \mid \left(L\right) \\
\end{array}
\]
We can assign the $\lambda$ construct a precedence level of 1 and the
application construct a precedence level of 2 (variables have a precedence
level of 3 but we don't need to give them an explicit construct).
\begin{verbatim}
\newcommand{\lam}[2]{
  \protectlevel{1}{
    \lambda #1.\requirelevel{1}{#2}}}

\newcommand{\app}[2]{
  \protectlevel{2}{
    \requirelevel{2}{#1}\,\requirelevel{3}{#2}}}
\end{verbatim}
We can now use these constructs to write arbitrary $\lambda$-calculus
expressions without worrying about parentheses.

\newcommand{\lam}[2]{
  \protectlevel{1}{
    \lambda #1.\requirelevel{1}{#2}}}

\newcommand{\app}[2]{
  \protectlevel{2}{
    \requirelevel{2}{#1}\,\requirelevel{3}{#2}}}

\begin{verbatim}
\newcommand{\deltaterm}{\lam{x}{\app{x}{x}}}
\newcommand{\omegaterm}{\app{\deltaterm}{\deltaterm}}
\end{verbatim}
\newcommand{\deltaterm}{\lam{x}{\app{x}{x}}}
\newcommand{\omegaterm}{\app{\deltaterm}{\deltaterm}}
\begin{verbatim}
\[
\Delta = \deltaterm
\]
\end{verbatim}
\[
\Delta = \deltaterm
\]
\begin{verbatim}
\[
\Omega = \app{\Delta}{\Delta} = \omegaterm
\]
\end{verbatim}
\[
\Omega = \app{\Delta}{\Delta} = \omegaterm
\]
\begin{verbatim}
\[
\app{\Omega}{\Omega} = \app{\omegaterm}{\omegaterm}
\]
\end{verbatim}
\[
\app{\Omega}{\Omega} = \app{\omegaterm}{\omegaterm}
\]

\end{document}

